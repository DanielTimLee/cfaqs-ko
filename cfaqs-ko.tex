% $Id$
\documentclass[10pt,a4paper,draft]{report}
\usepackage{graphicx}
\usepackage{epic}
\usepackage{hangul}
\usepackage{fancyheadings}

%
% In the new comp.lang.c FAQ, there are FAQ numbers which look like
% 12.15b.  I want to implement not to specify FAQ numbers in the `faq'
% environment, but I don't know how to implement \label command on the
% faq number like 12.15b.
%
% Do I have to implement not to increment faq counter?  If so,
% what about faq numbers look like 12.15c?
%
% Okay. I found the exact definition of \newcommand, and I solve this
% problem using optional argument to macros (Look the definition of \q.)
%
% But the real problem was not \q but \faqref.
% \faqref should generate a reference string looks like 12.5.
% In this example, I can generate the number 5 easily using \ref command,
% but I failed to generated the chapter number of the \label command.
%
% Perhaps I have to look the definition of \label and \ref command, but
% I don't know where.
%
% I decided to persuase The TeXbook by Donald E. Knuth so that I can
% understand the basic of TeX.
%
% Before that, I have to make \faqref accepts two argument; the chapter
% label and the faq label. (What a poor implementation it is!)
%

%
% ChangeLog
%
% 1.1.1.1 -- Import documents. (little modified from the first edition)
% 1.1.1.2 -- Define \page and \Page command and applied them.
% 

\newcommand{\page}[1]{\mbox{p.\ #1}}
\newcommand{\Page}[1]{\mbox{pp.\ #1}}

\newcommand{\initfaq}{\newcounter{faq}[chapter]}
%\newcommand{\faqref}[1]{\thechapter.\ref{#1}}
\newcommand{\faqref}[3][1]{\ref{#2}.\ref{#3}#1}

\newenvironment{faq}
	{\addtocounter{faq}{1}\vspace{1em}\begin{itemize}}{\end{itemize}}
\newenvironment{faq*}
	{\vspace{1em}\begin{itemize}}{\end{itemize}}
\newcommand{\Q}[1]{\item[\fbox{\textbf{\Large{Q #1}}}]}
%\newcommand{\q}{\item[\fbox{\textbf{\Large{\thechapter.\thefaq}}}]}
\newcommand{\q}[1][]{\item[\fbox{\textbf{\Large{\thechapter.\thefaq#1}}}]}
%\newcommand{\q}[1][]{\item[\fbox{\thechapter.\thefaq#1}]}

\newcommand{\A}{\item[Answer]}
\newcommand{\R}{\item[References]}
\newcommand{\T}{\item[Note]}
\newcommand{\seealso}[1]{���ٿ� ���� #1�� �����Ͻñ� �ٶ��ϴ�\@.}

\newcommand{\TT}[1]{\texttt{#1}}

\newcommand{\BF}[1]{\textbf{#1}}

\newcommand{\IT}[1]{\textit{#1}}
\newcommand{\EM}[1]{\textit{#1}}

\newcommand{\trans}[1]{(#1)}

\addtolength{\textwidth}{\marginparsep}
\addtolength{\textwidth}{\marginparwidth}
\addtolength{\textwidth}{\oddsidemargin}
\setlength{\oddsidemargin}{0cm}
\setlength{\evensidemargin}{0cm}
\setlength{\marginparsep}{0cm}
\setlength{\marginparwidth}{0cm}
\setlength{\voffset}{0cm}
\addtolength{\textheight}{1cm}


\title{C Programming FAQs}
\author{����: Seong-Kook Cin}
\date{\today}


\setlength{\parindent}{1ex}
\setlength{\parsep}{0.5\parsep}
\setlength{\parskip}{1ex}

\begin{document}

\����
\pagestyle{fancy}

\maketitle

\vspace{10em}
\begin{center}
\EM{This book is decicated to C programmers everywhere.}

\vspace{2cm}

\BF{��� C ���α׷��ӵ鿡�� �� å�� ��Ĩ�ϴ�.}
\end{center}

\newpage
\tableofcontents



% chapter: preface
\newpage
\input preface

% chapter 1: decinit
\newpage
\input decinit

% chapter 2: struct
\newpage
\input struct

% chapter 3: expr
\newpage
\input expr

% chapter 4: pointer
\newpage
\input pointer

% chapter 5: nullptr
\newpage
\input nullptr

% chapter 6: arrayptr
\newpage
\input arrayptr

% chapter 7: memalloc
\newpage
\input memalloc

% chapter 8: charstr
\newpage
\input charstr

% chapter 9: boolexpr
\newpage
\input boolexpr

% chapter 10: preproc
\newpage
\input preproc

% chapter 11: standard
\newpage
\input standard

% chapter 12: stdio
\newpage
\input stdio

% chapter 13: libfunc
\newpage
\input libfunc

% chapter 14: float
\newpage
\input float

% chapter 15: vlal
\newpage
\input vlal

% chapter 16: sproblem
\newpage
\input sproblem

% chapter 17: style
\newpage
\input style

% chapter 18: toolres
\newpage
\input toolres

% chapter 19: sysdep
\newpage
\input sysdep

% chapter 20: misc
\newpage
\input misc

% chapter 21: ext
\newpage
\input ext

% bibliography
\newpage
\input bib

\end{document}


% $Id$
\chapter{Characters and Strings}	\label{chap:charstr}
% chapter 8: charstr

\begin{rawhtml}
<pre>$Id$</pre>
\end{rawhtml}

C has no built-in string type; by convention, a string is represented as an
array of characters terminated with '\verb+\0+'.  Furthermore, C hardly has
a character type; a character is represented by its integer value in the
machine's character set.  Because these representations are laid bare and are
visible to C programs, programs have a tremendous amount of control over how
characters and strings are manipulated.  The downside is that to some extent,
programs \EM{have} to exert this control:  The programmer must remember
whether a small integer is being interpreted as a numeric value or as a
character (see question \ql{8.6}) and must remember to maintain arrays (and
allocated blocks of memory) containing strings correctly.

See also question \ql{13.1} through \ql{13.7}, which cover library functions for
string handling.

\begin{faq}
\Q{8.1} % 8.1 COMPLETE
	왜 다음 코드는 실행이 안되는 것일까요?
\begin{verbatim}
  strcat(string, '!');
\end{verbatim}
\A	
	문자열과 문자는 큰 차이가 있습니다.  \TT{strcat()} 함수는 문자열을
	이어주는 함수입니다.
        
        \verb+'!'+와 같은 문자 상수(character constant)는 한 문자를 나타냅니다.
        문자열(string literal)은 쌍따옴표로 둘러싼, 대개 여러 글자의 문자로
        구성됩니다. \verb+"!"+와 같은 문자열은 하나의 문자를 나타내는 것처럼
        보이지만, 실제로 \verb+!+ 문자와, 문자열 끝을 나타내는 \verb+\0+, 두
        개의 문자로 구성됩니다.

	C 언어에서 문자는 문자 집합(character set)에서 그 문자가 나타내는
	작은 정수 값으로 표현됩니다 (질문 \ql{8.6}을 참고하기 바랍니다).
	문자열은 문자들의 배열로서 표현됩니다; 문자열을 다룰 때에는
	보통 문자 배열의 첫 요소를 가리키는 포인터를 써서 합니다.
% 다음 문장 번역,
        It is never correct to use one when the other is expected.
	문자열에 \verb+!+를 이어 붙이려면 다음과 같이 해야 합니다.
\begin{verbatim}
  strcat(string, "!");
\end{verbatim}

	질문 \ql{1.32}, \ql{7.2}, \ql{16.6}을 참고하기 바랍니다.
\R
	\cite{ctp} \S\ 1.5 \Page{9--10}
\end{faq}

\begin{faq}
\Q{8.2} % 8.2 COMPLETE
	문자열이 어떤 특정한 값과 같은지 검사하려고 합니다.  다음과 같이
	코드를 만들었는데 왜 동작하지 않을까요?
\begin{verbatim}
  char *string;
  ...
  if (string == "value") {
    /* string matches "value" */
    ...
  }
\end{verbatim}
\A
	C 언어는 문자열을 문자의 배열로 처리합니다.  그리고 C 언어에서는
	배열 전체에 대해 어떤 연산을 (대입, 비교 등) 직접 할 수 있는
	방법은 없습니다.  위의 코드에서 \verb+==+ 연산은
	피연산자인 포인터의 값을 비교합니다 --- 즉 변수 \TT{string}의
	포인터 값과 문자열 \verb+"value"+를 가리키는 포인터 값을 
        비교합니다 ---
	따라서 두 개의 포인터가 같은 곳을 가리키는지를 비교합니다.
	대개의 경우 이 값이 같게 될 경우는 거의 없으므로, 이 비교는
	거의 항상 같지 않다고 나옵니다.

	두 문자열을 비교하는 방법으로 라이브러리 함수인 \TT{strcmp()}를 쓰는
	것이 가장 좋습니다:

\begin{verbatim}
  if (strcmp(string, "value") == 0) {
    /* string matches "value" */
    ...
  }
\end{verbatim}
\end{faq}

\begin{faq}
\Q{8.3} % 8.3 COMPLETE
	다음과 같이 할 수 있다면:
\begin{verbatim}
  char a[] = "Hello, world!";
\end{verbatim}
	\noindent 왜 이렇게는 할 수 없을까요?
\begin{verbatim}
  char a[14];
  a = "Hello, world!";
\end{verbatim}
\A	문자열은 배열입니다.  그리고 배열에는 직접 대입 연산을 쓸 수
	없습니다.  \TT{strcpy()} 함수를 쓰기 바랍니다:
\begin{verbatim}
  strcpy(a, "Hello, world!");
\end{verbatim}
	질문 \ql{1.32}, \ql{4.2}, \ql{6.5}, \ql{7.2}를 참고하기 바랍니다.
\end{faq}

\begin{faq}
\Q{8.4} % 8.4 COMPLETE
	왜 \TT{strcat}이 동작하지 않을까요? 아래 코드처럼 했는데, 잘
        안됩니다:
\begin{verbatim}
  char *s1 = "Hello, ";
  char *s2 = "world!";
  char *s3 = strcat(s1, s2);
\end{verbatim}
\A
	질문 \ql{7.2}를 보기 바랍니다.
\end{faq}

\begin{faq}
\Q{8.5} % 8.5 COMPLETE
	아래 두 초기화 문장의 차이가 있나요?
\begin{verbatim}
  char a[] = "string literal";
  char *p = "string literal";
\end{verbatim}
	\noindent 제 프로그램에서 \TT{p[i]}에 어떤 값을 대입하려 하면
        프로그램이 비정상적으로 끝나버립니다.
\A
	질문 \ql{1.32}를 보기 바랍니다.
\end{faq}

\begin{faq}
\Q{8.6} % 8.6 COMPLETE
	문자에 해당하는 수치 (ASCII 또는 다른 문자 set code)) 값을
        어떻게 얻을 수 있죠?
	또 그 반대로 수치 값에서 그 값에 해당하는 문자를 어떻게 얻을 수 있죠?

\A
	C 언어에서 문자는 (컴퓨터의 문자 셋의) 문자 코드 번호로 작은
	정수로서 표현됩니다.  따라서 질문한 것과 같은 변환이 필요없습니다;
	즉 문자를 가지고 있다면, 그 자체가 값이 됩니다. 예를 들어, 다음 코드는:
\begin{verbatim}
  int c1 = 'A', c2 = 65;
  printf("%c %d %c %d\n", c1, c1, c2, c2);
\end{verbatim}
	ASCII를 쓰는 시스템에서, 다음과 같은 출력을 만들어 냅니다:
\begin{verbatim}
  A 65 A 65
\end{verbatim}
	\noindent \seealso{\ql{8.9}, \ql{20.10}}
\end{faq}

\begin{faq}
\Q{8.7} % 8.7 COMPLETE
	C 언어에서, 다른 언어에서 제공하는 것처럼, 문자열의 일부를 뽑아내는,
        ``substr''와 같은 기능이 있나요?
\A
	질문 \ql{13.3}를 보기 바랍니다.
\end{faq}

\begin{faq}
\Q{8.8} % 8.8 COMPLETE
	사용자가 입력한 문자열을 읽어서 배열에 저장한 다음, 나중에 출력하려고
        합니다. 사용자가 \verb+\n+와 같은 문자를 입력한 경우, 왜 제대로
        처리되지 않을까요?
\A
	\verb+\n+와 같은 문자 시퀀스(character sequence)들은 컴파일할 때
        해석됩니다. 문자 상수나 문자열에 백슬래시가 나오고 바로 뒤에 
        \verb+n+이 나오면, 한 글자, newline 문자로 해석됩니다. 
        (다른 character escape sequence도 비슷한 방법으로 처리됩니다.)
        사용자나 파일에서 문자열을 읽을 때는 이와 같은 해석이 적용되지 않습니다:
        즉, 백슬래시는 다른 문자와 전혀 다를게 없이 취급되며, 하나의 문자로
        간주됩니다. (run-time I/O가 일어날 때, newline 문자를 위해 어떤
        번역 작업이 이뤄질 수 있지만, 이 것은 전혀 다른 문제입니다. 질문
        \ql{12.40}을 보기 바랍니다.) \seealso{\ql{12.6}}
\end{faq}

\begin{faq}
\Q{8.9} % 8.9 COMPLETE
	제 컴파일러에 버그가 있습니다.  \verb+sizeof('a')+의
        값이 \verb+sizeof(char)+인 1로 나오지 않고, 2가 나옵니다.
\A
	놀랍게도, C 언어에서 문자 상수(character constant)의
        타입은 \TT{int}입니다.
	따라서 \verb+sizeof('a')+는 \verb+sizeof(int)+와 같습니다.
        (C++에서는 조금 다릅니다.) \seealso{\ql{7.8}}
\T
	참고로 C++에서 문자 상수의 타입은 \TT{char}입니다. 즉,
        \verb+sizeof('a')+는 \verb+sizeof(char)+와 같습니다.
\R
	\cite{ansi} \S\ 3.1.3.4 \\
	\cite{c89} \S\ 6.1.3.4 \\
	\cite{hs} \S\ 2.7.3 \page{29}
\end{faq}

\begin{faq}
\Q{8.10} % 8.10
	I'm starting to think about multinational character sets. Should I
        worry about the implication of making \TT{sizeof(char)} be 2 so that
        16-bit character sets can be represented?
%	여러 언어를 처리할 수 있는 문자 세트(multinational character set)를
%        생각하고 있습니다. 16-bit 문자 세트를 지원하기 위해, 
%        \verb+sizeof(char)+가 2가 되는 것을 기대할 수 있을까요?
% TODO: 맞게 번역한 건가?
\A
	만약 \TT{char}가 16 bit가 된다고 해도, \TT{sizeof(char)}는 여전히
        1이 됩니다. 대신 \verb+<limits.h>+에 정의된, \verb+CHAR_BIT+이
        16이 됩니다. 이 경우에는 한, 8-bit 오브젝트를 선언하는 것은 (또,
        \TT{malloc}으로 할당하는 것도) 불가능합니다.

        전통적으로, 한 바이트가 꼭 8 bit일 필요는 없습니다. 단지, 한 글자를
        저장하기에 충분한, 작은 크기의 메모리 공간이면 됩니다. C 표준에서도
        이 방식을 따르며, 따라서 \TT{malloc}이나 \TT{sizeof}에서 쓰이는
        바이트가 8 bit 이상일 수도 있습니다.\footnote{전문 용어로, 8 bit로
          표현되는 바이트를 ``octet''이라고 합니다.} (대신 표준에서, 바이트가
        8 bit 이상되어야 한다고 정해 놓았습니다.)

        Multinational character set을 처리하기 위해서는 \TT{char} 이상의
        어떤 타입이 필요하고, ANSI/ISO C 표준에서는 이를 위해, 더 큰 범위를
        포함할 수 있다는 뜻의 ``wide'' 문자 타입인, \verb+wchar_t+를
        제공합니다. 또한 이 타입으로 처리하는 wide 문자 상수, wide 문자열,
        또 wide 문자 및 문자열을 처리할 수 있는 함수들을 제공합니다.

        \seealso{\ql{7.8}}
\R
	\cite{ansi} \S\ 2.2.1.2, \S\ 3.1.3.4, \S\ 3.1.4, \S\ 4.1.5, \S\ 4.10.7,
                    \S\ 4.10.8 \\
        \cite{c89} \S\ 5.2.1.2, \S\ 6.1.3.4, \S\ 6.1.4, \S\ 7.1.6, \S\ 7.10.7,
                   \S\ 7.10.8 \\
        \cite{rationale} \S\ 2.2.1.2 \\
        \cite{hs} \S\ 2.7.3 \Page{29--30}, \S\ 2.7.4 \page{33}, 
                  \S\ 11.1 \page{293}, \S\S\ 11.7, 11.8 \Page{303--10}
\end{faq}

%
% Local Variables:
% coding: utf-8
% fill-column: 78
% End:
%

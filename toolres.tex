\chapter{Tools and Resources}	\label{chap:toolres}
% chapter 18: toolres

\begin{faq}
\Q{18.1}
	도움이 될만한 프로그램이 필요합니다.
\A
	아래의 테이블을 참고하기 바랍니다 (\seealso{\ql{18.16}}):

\small
\begin{tabular}{l|p{6cm}} \hline \hline
cross-reference generator	
	& cflow, cxref, calls, cscope, xscope, ixfw \\ \hline
C beautifier/pretty printer 
	& cb, indent, GNU indent, vgrind \\ \hline
버전/설정 관리 
	& CVS, RCS, SCCS \\ \hline
C source obfuscator 
	& obfus, shroud, opqcp \\ \hline
``make'' dependency generator
	& makedepend, \TT{cc -M}, \TT{cpp -M} \\ \hline
compute code metrics
	& ccount, Metre, lcount, csize, \\ \hline
C lines-of-source counter 
	& wc, \TT{grep -c ";"} \\ \hline
C declaration aid 
	& \TT{comp.sources.unix}의 Volumn 14.  \break \cite{kr2} 참고 \\ \hline
tracking down \TT{malloc} problem & 질문 \ql{18.2} 참고 \\ \hline
``selective'' C preprocessor & 질문 \ql{10.18} 참고 \\ \hline
language translation tool & 질문 \ql{11.31} 참고 \\ \hline
C verifier (lint) & 질문 \ql{18.7} 참고 \\ \hline
C compiler & 질문 \ql{18.3} 참고 \\ \hline \hline
\end{tabular}
\normalsize

	(당연히 이 테이블이 모든 것을 나타내 주는 것은 아닙니다;
	만약 여기에 나온 것 이외에 더 많은 것을 알고 있다면
	관리자에게 연락하기 바랍니다.)

	Usenet \TT{comp.compilers}와
	\TT{comp.software-eng}에서 여러 툴에 대한 다른 목록과, 논쟁을 
        찾아 볼 수 있습니다.

	\seealso{\ql{18.3}, \ql{18.16}}
\T
	위 테이블의 code metric에 관한 것은 다음 URL을 참고하시기 바랍니다:

\begin{verbatim}
  http://www.qucis.queensu.ca/Software-Engineering/Cmetrics.html
\end{verbatim}

	\noindent GNU indent에 관한 것은 아래 URL을 참고하시기 바랍니다:
\begin{verbatim}
  http://www.gnu.org/software/indent/
\end{verbatim}

\end{faq}

\begin{faq}
\Q{18.2}
	\TT{malloc}에서 문제가 발생한 것 같은데 검사를 쉽게 해주는 
	툴은 없나요?

\A
	\TT{malloc} 문제를 도와주는 여러가지 디버깅 패키지들이 있습니다.
	인기있는 것 중의 하나는 conor p.\ Cahill씨의 ``dbmalloc''이며,
	1992년 \TT{comp.sources.misc}의 volumn 32에서 구할 수 있습니다.
	또 \TT{comp.sources.unix} volumn 27에서 얻을 수 있는 ``leak''도
	좋습니다; 
	JMalloc.c and JMalloc.h in the "Snippets" collection;
	and MEMDEBUG from ftp.crpht.lu in pub/sources/memdebug .   See
	also question \ql{18.16}.

	A number of commercial debugging tools exist, and can be
	invaluable in tracking down malloc-related and other stubborn
	problems:

	\begin{itemize}
	\item
		Bounds-Checker for DOS, from Nu-Mega Technologies,
		P.O.  Box 7780, Nashua, NH 03060-7780, USA, 603-889-2386.

	\item
		CodeCenter (formerly Saber-C) from Centerline Software,
		10 Fawcett Street, Cambridge, MA 02138, USA, 617-498-3000.

	\item
		Insight, from ParaSoft Corporation, 2500 E.  Foothill
		Blvd., Pasadena, CA 91107, USA, 818-792-9941,
		insight@parasoft.com .

	\item
		Purify, from Pure Software, 1309 S.  Mary Ave., Sunnyvale,
		CA 94087, USA, 800-224-7873, http://www.pure.com ,
		info-home@pure.com .
		(I believe Pure was recently acquired by Rational.)

	\item
		Final Exam Memory Advisor, from PLATINUM Technology
		(formerly Sentinel from AIB Software), 1815 South Meyers
		Rd., Oakbrook Terrace, IL 60181, USA, 630-620-5000,
		800-442-6861, info@platinum.com, www.platinum.com .

	\item
		ZeroFault, from The Kernel Group, 1250 Capital of Texas
		Highway South, Building Three, Suite 601, Austin,
		TX 78746, 512-433-3333, http://www.tkg.com, zf@tkg.com .
	\end{itemize}

\T
	요즈음에는 다음과 같은 툴들이 인기가 많습니다:
        \begin{itemize}
          \item Electric Fence (또는 줄여서 efence) --- C 프로그램에서
            buffer overrun과 underrun을 검사해 줍니다.
          \item Valgrind --- 다양한 메모리 에러를 검사해 주며, 코드를
            다시 컴파일할 필요가 없습니다. 다만 현재 x86 프로세스를 쓰는
            Linux 시스템에서만 쓸 수 있습니다.
        \end{itemize}
\end{faq}

\begin{faq}
\Q{18.3}
	저렴하거나 공짜로 구할 수 있는 컴파일러가 있을까요?
\A
	인기있고, 공짜로 구할 수 있으며, 고 품질인 FSF의 GNU C 컴파일러
	(또는 gcc라고 불리움)를 쓰면 됩니다.  이는 \TT{prep.ai.mit.edu}의
	\TT{pub/gnu} 디렉토리나 기타 GNU 아카이브 사이트에서 구할 수 있습니다.
	MS-DOS 용으로 포팅한 djgpp도 있으며,
	\TT{http://www.delorie.com/djgpp/}에서 얻을 수 있습니다.

	PCC라는 쉐어웨어(shareware) 컴파일러도 있으며, 이름은
	\TT{PCC12C.ZIP}입니다.

	MS-DOS용 컴파일러로 매우 싼 가격에 구할 수 있는 Power C는
	Mix Software에서 만들었고, 주소는 1132 Commerce Drive, Richardson,
	TX 75801, USA, 전화번호는 214-783-6001입니다.

	최근에 개발된 컴파일러로는 lcc가 있습니다.  Anonymous FTP로 \\
	\TT{ftp.cs.princeton.edu}의 \TT{pub/lcc}에서 구할 수 있습니다.

	\TT{ftp.hitech.com.au}의 \TT{hitech/pacific}에서 쉐어웨어 MS-DOS용
	컴파일러를 구할 수 있습니다.  사고파는 목적이 아니라면 등록은 할 필요가
	없습니다.

	매킨토시 용으로 구할 수 있는 쉐어웨어 컴파일러는 알려진 것이 없습니다.

	\TT{comp.compilers} archive에는 compiler, interpreter,
        grammer등에 대한 굉장히 많은 정보가 들어 있습니다.
	(FAQ 리스트를 포함한) \TT{comp.compilers} archive는 iecc.com에서
	근무하는 moderator인 John R.  Levine씨에 의해 관리되고 있습니다.
	사용 가능한 컴파일러들과 이에 관련된 자료는 Mark Hopkins, Steven
	Robenalt, 그리고 David Muir Sharnoff씨에 의해 관리되며
	\TT{ftp.idiom.com}의 \TT{pub/compilers-list/}에 있습니다.
	(\TT{rtfm.mit.edu} 또는 \TT{ftp.uu.net}의 \TT{comp.compilers}
	디렉토리에서 \TT{news.answers}에 관한 아카이브도 참고하기 바랍니다;
	질문 \ql{20.40}도 참고하기 바랍니다.)

\T
	위 답변은 상대적으로 오래된 것입니다.   GCC는 원래 ``GNU C Compiler''의
	약자이었지만,  C 언어 뿐만 아니라 C++, java와 같은 언어도 지원하기
	때문에 이름이 ``GNU Compiler Collection''으로 이름이 바뀌었습니다.
	GCC와 더불어 GNU software를 구하시려면 \TT{ftp.gnu.org}의
	\TT{/pub/}에서
	구할 수 있으며, 국내에서는 \TT{ftp.bora.net}의 \TT{/pub/gnu/}에서
	download 받는 것이 빠릅니다.   자세한 것은 GCC의 홈페이지인
        아래 사이트를 방문하시면 얻을 수 있습니다:
\begin{verbatim}
  http://www.gnu.org/software/gcc/
\end{verbatim}

	\TT{lcc}는 표준 C 언어로 씌여진, 이식성이 매우 우수한(retargetable)
        C 컴파일러이며, ALPHA, SPARC, MIPS R3000, Intel x86용 코드를 만들어
        냅니다. \TT{lcc}는 소스가 다른 컴파일러에 비해 상대적으로 짧고 읽기가
        쉬워서 컴파일러 교육용으로 많이 쓰입니다. 아래 사이트를 방문하시면
        좀 더 많은 정보를 얻을 수 있습니다:
\begin{verbatim}
  http://www.cs.princeton.edu/software/lcc/
\end{verbatim}

% TODO: cygwin(cygnus?) toolkit 설명.

	\seealso{\ql{18.16}}
\end{faq}

\begin{faq}
\Q{18.4}
	프로그램을 동작시켰는데, 매우 이상하게 동작합니다.
	잘못된 점을 어떻게 찾을 수 있죠?

\A
	먼저 lint를 돌려보시기 바랍니다.  (아마도 \TT{-a}, \TT{-c}, \TT{-h},
	\TT{-p} 등의 옵션을 함께 써야 할 것입니다.)
        대부분의 C 컴파일러는 완벽한 컴파일러가 아닙니다. 다른 말로,
        만약 프로그래머가, 할 일을 명확하게 지시하지 않았거나, 지시한 사항이
        동작한다고 보장할 수 없는 경우에 무시하고 넘어가는 경우가 많습니다.
        (그런 경우에도, 컴파일러가 제공하는 경고 메시지를 최대로 출력하도록
        해 보기 바랍니다.)

	\seealso{\ql{16.5}, \ql{16.8}, \ql{18.7}}

\R
	\cite{lint}
\end{faq}

\begin{faq}
\Q{18.5}
	`lint'를 실행시키면 \TT{malloc()}을 부를 때마다 ``warning: possible 
	pointer alignment problem''라는 경고가 발생하는데, 어떻게 이 경고좀
	발생하지 않게 할 수 없을까요?

\A
	오래된 lint의 경우, \TT{malloc()}이 ``어떠한 타입의 object도 저장할
	수 있는 공간을 가리키는 포인터를 리턴한다''라는 사실을 모르기
	때문에 그런 경고가 나옵니다.
	한가지 방법은 \verb+#ifdef lint+ 안에 \verb+#define+을 써서
	\TT{malloc()}에 대한 `pseudo-implementation'을 만들어 두는 것입니다.
	그렇지만 정말로 중요한 경고 메시지도 보여주지 않을 가능성이 있기 때문에
	좋은 방법이라 할 수는 없습니다.  차라리 `grep -v'를 써서 그런 메시지를
	없애버리는 것이 훨씬 더 간단하고 안전할 것입니다.
	(그러나 `lint'가 보여주는 많은 메시지들을 무시하는 습관을 들이는 것은
	매우 위험합니다.  어느날 중요한 메시지를 못 보고 넘어가는 경우도
	생길지 모릅니다.)
\end{faq}

\begin{faq}
\Q{18.7}
	ANSI 호환의 lint를 구할 수 있을까요?
\A
	대부분 시스템에서 쓸 수 있는 `PC-Lint'와 `FlexeLint'가 있으며,
	아래 주소에서 구할 수 있습니다:
\begin{verbatim}
  Gimpel Software
  3207 Hogarth Lane
  Collegeville, PA  19426  USA
  (+1) 610 584 4261
  gimpel@netaxs.com
\end{verbatim}

	Unix System V release 4의 lint는 ANSI 호환입니다.  그리고
	UNIX Support Labs나 System V 판매업체에 의해 각각 다른 패키지
	형태 (다른 C tool들이랑 같이 번들 형태)로 제공됩니다.

	고수준으로 문법을 검사해주는 ANSI 호환의 다른 lint로는
	`LCLint'가 있으며, \TT{larch.lcs.mit.edu}의 \TT{pub/Larch/lclint}에서
	구할 수 있습니다 (via anonymous ftp).

	그리고 최근에 제공되는 많은 컴파일러들은 lint가 제공하는 것처럼
	자세히 소스를 분석하고 알려주는 기능이 있습니다.  (`gcc'의 경우는
	`-Wall -pedantic' 옵션을 쓰면 됩니다.)

\T
	lclint는 GPL\footnote{GNU General Public License}로 
	라이센스되어 있으며, 홈페이지는 다음과 같습니다:
\begin{verbatim}
  http://lclint.cs.virginia.edu/
\end{verbatim}

	2002년 이후, lclint는 splint로 이름을 바꾸었으며, 홈페이지는
        다음과 같습니다:
\begin{verbatim}
  http://www.splint.org/
\end{verbatim}
\end{faq}

\begin{faq}
\Q{18.8}
	ANSI function prototype이 있으니깐 이제 lint는 필요없지 않나요?
\A

	아닙니다.  무엇보다도 prototype은 존재하고 있고, 올바로 주어질 때에만
	동작합니다; 잘못된 prototype은 아예 없는 것이 낫습니다.  두번째로
	lint는 여러 소스 파일들을 읽고 일관성을 검사해 줄 수 있고,
	함수뿐만 아니라 데이터 선언까지 검사해 줍니다.
	게다가 lint와 같이 시스템에 독립적인 프로그램을 사용하면 좀더
	정확한 분석을 받을 수 있기 때문에, 컴파일러의 특별한 기능이나,
	특정 시스템 기능에 의존하지 않는, 이식성이 높은 프로그램을 작성하는데
	큰 도움이 됩니다.

	`lint'를 쓰는 대신 function prototype을 쓰길 원한다면,
	먼저 여러 파일들이 같은 prototype을 지녔는지 검사하고,
	정확한 prototype을 썼는지 검사해야 합니다.  질문 \ql{1.7}과 \ql{10.6}을
	참고하기 바랍니다.
\end{faq}

\begin{faq}
\Q{18.9}
	인터넷상에서 구할 수 있는 C 언어 안내서나 기타 자료는 없나요?

\A
	물론 많습니다:

	\begin{itemize}
	\item Tom Torfs씨는 매우 좋은 튜토리얼을 가지고 있습니다.  \break
	\TT{http://members.xoom.com/tomtorfs/cintro.html}.

	\item Christopher Sawtell씨가 만든 ``Notes for C programmers''는
	아래 주소에서 구할 수 있습니다:
\begin{itemize}
\item \TT{svr-ftp.eng.cam.ac.uk}의 \TT{misc/sawtell\_C.shar}
\item \TT{garbo.uwasa.fi}의 \TT{/pc/c-lang/c-lesson.zip}
\end{itemize}

	\item Tim Love씨의 ``C for Programmers''는 \TT{svr-ftp.eng.cam.ac.uk}의
	\TT{misc} 디렉토리에서 구할 수 있습니다.  HTML 버전은 아래에서
	구할 수 있습니다:
\begin{verbatim}
  http://www-h.eng.cam.ac.uk/help/tpl/languages/C/teaching_C/
\end{verbatim}

	\item Coronado Enterprises C tutorial은 Simtel 미러 사이트의
	\TT{pub/msdos/c} 또는 \TT{http://www.swcp.com/~dodrill}에서
	구할 수 있습니다.

	\item Rick Rowe씨의 tutorial은 \TT{ftp.netcom.com}의 
	\TT{pub/rowe/tutorde.zip} 또는 \TT{ftp.wustl.edu}의
	\TT{pub/MSDOS\_UPLOADS/programming/c\_language/ctutorde.zip}에서
	구할 수 있습니다.

	\item Web용으로 만들어진 코스로는:
\begin{verbatim}
  http://www.strath.ac.uk/CC/Courses/CCourse/CCourse.html
\end{verbatim}

	\item Martin Brown씨는 C 코스 자료를 Web에 게시했습니다:
\begin{verbatim}
  http://www-isis.ecs.soton.ac.uk/computing/c/Welcome.html
\end{verbatim}

	\item 어떤 Unix 시스템에서는 \TT{learn c}라고 shell prompt에서
	입력하면 간단하게 가르쳐 줍니다 (그러나 상당히 오래된 내용일 것입니다).

	\item 마지막으로 이 FAQ 목록의 저자도 C 언어 교육을 하고 있으며,
	그 노트를 web에서 구할 수 있습니다;
\begin{verbatim}
  http://www.eskimo.com/~scs/cclass/cclass.html
\end{verbatim}

	[Disclaimer: 여기에 있는 많은 tutorial들은 저자가 다 검토해 본 것은
	아닙니다.  또한 몇 가지 에러가 있는 것으로 압니다.  이 글들에 저자의
	이름이 들어 있기는 하지만 어떠한 것도 보증할 수 없습니다.
	또한 이런 종류의 글들은 대개 시간이 지날 수록 값어치가 떨어지기
	때문에 여러분이 읽으려고 할 때에는 이미 너무 구식의 글일지도
	모릅니다.]

	\item 이러한 tutorial들과 C 언어에 관한 여러가지 정보는
	아래 URL에서 구할 수 있습니다:
\begin{verbatim}
  http://www.lysator.liu.se/c/index.html
\end{verbatim}

	\item Vinit Carpenter씨는 C와 C++을 배울때 쓸모있는 여러가지 자료에
	대한 목록을 유지하고 있습니다.  이 목록은 newsgroup \TT{comp.lang.c}와
 	\TT{comp.lang.c++}에 게시되며 이 FAQ list가 있는 곳에 (질문 \ql{20.40}
	참고), 같이 보관되어 있으며, 아래에서 구할 수 있습니다.
\begin{verbatim}
  http://www.cyberdiem.com/vin/learn.html
\end{verbatim}
	\end{itemize}

	\seealso{\ql{18.10}, \ql{18.15c}}

\T
	원 저자의 홈페이지의 다음 링크에서 관련된 자료를 찾아볼 수 있습니다:
\begin{verbatim}
  http://www.eskimo.com/~scs/C.html
\end{verbatim}


\end{faq}

\begin{faq}
\Q{18.10}
	C 언어를 배우려고 하는데 좋은 책좀 추천해 주세요.

\A	
	C 언어에 관한 책들은 이 글에서 모두 다루기에는 너무나도 많습니다;
	따라서 이들 책 모두를 평가해본다는 것은 너무나도 힘이 듭니다.
	많은 사람들이 가장 처음 만들어진 책이 가장 좋다고 말하며
	\EM{The C Programming Language}를 추천합니다.  이 책은 Kernighan씨와
	Ritchie씨가 썼습니다 (``K\&R''이라고도 하며, 현재 두번째 판이
	나와 있습니다).   그러나 K\&R이 초보자가 보기에는 적당하지 않다는
	의견도 있습니다; 그렇지만 우리들 대부분이 그 책으로 C를 배웠으며,
	아주 잘 배웠다고 생각합니다; 그러나 책의 내용이 너무나 딱딱하기
	때문에 프로그래밍에 대한 배경 지식이 전혀 없는 사람에게는
	무리일 것입니다.   이 책에 대한 여러 보충 설명이나 정정 목록은
	아래에서 구할 수 있습니다:

	\begin{itemize}
	\item \TT{http://www.csd.uwo.ca/~jamie/.Refs/.Footnotes/C-annotes.html}
	\item \TT{http://www.eskimo.com/~scs/cclass/cclass.html}
	\item \TT{http://www.lysator.liu.se/c/c-errata.html\#main}
	\end{itemize}

	\TT{comp.lang.c}의 많은 사람들은 K.N.  King씨가 쓴
	\EM{C: A Modern approach}를 추천합니다.

	가장 좋은 참고 서적은 \EM{C: A Reference Manual}이며,
	Samuel p.\ Harbison씨와 Guy.  L.  Steele씨가 썼습니다.  현재
	네번째 판이 나와 있습니다.

	C 언어를 처음부터 배우기에는 적당하지 않지만 이 FAQ 목록도
	출판되어 책으로 나와 있습니다. 
	이 글의 마지막 ``저서 목록''의 \cite{cfaqs}를 참고하시기 바랍니다.

	Mitch Wright씨는 C와 UNIX에 관한 많은 참고서에 대한 목록을 가지고
	있으며, \TT{ftp.rahul.net}의 \TT{pub/mitch/YABL}에서 구할 수 있습니다.

	Scott McMahon씨는 여러가지 비평에 대한 글을 가지고 있습니다:
\begin{verbatim}
  http://www.skwc.com/essent/cyberreviews.html
\end{verbatim}

	`Association of C and C++ Users (ACCU)'는 C/C++에 대한 
	많은 책들에 대한 논평을 모아 놓았습니다:
\begin{verbatim}
  http://bach.cis.temple.edu/accu/bookcase
  http://www.accu.org/accu
\end{verbatim}

	이 FAQ 목록의 편집자도 권장하는 많은 책에 대한 정보를 가지고
	있습니다; 필요하면 보내드리고 있습니다.  \seealso{\ql{18.9}}

\T
	\EM{C: A Reference Manual}은 현재 다섯번째 판이 나와 있습니다.
        자세한 것은 다음 사이트를 참고하기 바랍니다:
\begin{verbatim}
  http://www.careferencemanual.com/
\end{verbatim}

% TODO: 내 기준으로 좋은 책들을 여기에 추천하기
% 먼저 expert c, c traps and fitfall, ...
\end{faq}

\begin{faq}
\Q{18.13}
	표준 C 라이브러리에 대한 소스를 구할 수 있을까요?

\A
	(public domain은 아니지만) P.J.  Pluager씨의
	\EM{The Standard C Library}가 있습니다.
	모든 C 라이브러리에 대한 소스 코드는 NetBSD나
	GNU (물론 Linux도) project의 일부분으로 제공됩니다.
	\seealso{\ql{18.15c}, \ql{18.16}}

\T
	GNU C 라이브러리의 홈페이지는 아래와 같습니다:
\begin{verbatim}
  http://www.gnu.org/software/glibc/
\end{verbatim}

\end{faq}

\begin{faq}
\Q{18.13b}
	온라인 C reference manual이 있을까요?
\A
	두 개가 있습니다:
	\begin{itemize}
	\item \TT{http://www.cs.man.ac.uk/standard\_c/\_index.html}
	\item \TT{http://www.dinkumware.com/htm\_cl/index.html}.
	\end{itemize}
\end{faq}

\begin{faq}
\Q{18.13c}
	ANSI/\cite{c89} C 표준을 구할 수 있을까요?
\A
	질문 \ql{11.2}를 보기 바랍니다.
\end{faq}

\begin{faq}
\Q{18.14}
	수식(expression)을 파싱(parsing)하고 평가하는 코드가 필요합니다.
\A

	두 개의 패키지가 각각 1993년 12월에 \TT{comp.sources.misc}에 
	1994년 1월에 \TT{alt.sources}에 게시되었지만, 지금은 없고,
	대신 \TT{sunsite.unc.edu}의 다음 위치에서 얻을 수 있습니다:
\begin{verbatim}
  pub/packages/development/libraries/defunc-1.3.tar.Z
\end{verbatim}
	또, \TT{lamont.ldgo.columbia.edu}에서 ``parse'' 패키지도
        찾아보기 바랍니다.
	여러가지 옵션들이 포함되어 있는 S-Lang 인터프리터는 anonymous ftp로
	\TT{amy.tch.harvard.edu}의 \TT{/pub/slang}에서 구할 수 있습니다.
	또한 Cmm (``C-minus-minus'')이라는 쉐어웨어도 있습니다.
	\seealso{\ql{18.16}, \ql{20.6}}

	\EM{Software Solutions in C}의
	Chapter~\ref{chap:stdio},
        \mbox{페이지 235--55}에서 파싱/평가하는 코드를
        찾아볼 수 있습니다.
\end{faq}

\begin{faq}
\Q{18.15}
	C 언어에 대한 BNF 또는 YACC 문법을 구할 수 있을까요?
\A
	완벽한 문법은 물론 ANSI 표준에 들어 있습니다; 질문 \ql{11.2}를
	보기 바랍니다.  다른 문법으로는 (C++과 관련된) Jim Roskind씨가
	만든 것인데 \TT{ics.uci.edu}의 \TT{pub/c++grammar1.1.tar.Z}로
	얻을 수 있습니다.  아마도 \TT{ftp.ics.uci.edu},
	\TT{OLD/pub/c++grammar1.1.tar.Z}일지도 모릅니다.  또는
	\TT{ftp.eskimo.com}의 \TT{u/s/scs/roskind\_grammar.Z}로 구할 수
	있습니다.
	작업중인 ANSI 문법은 (Jeff Lee씨가 작업하는) \TT{ftp.uu.net}의
	(질문 \ql{18.16} 참고) \TT{usenet/net.sources/ansi.c.grammer.Z}에서
	구할 수 있으며, lexer가 포함되어 있습니다.
	\cite{kr2}의 부록이나 FSF의 GNU C compiler에서도 문법을 얻을 수 있습니다.

	\TT{comp.compilers} 아카이브는 문법에 대한 많은 정보를 가지고
	있습니다; 질문 \ql{18.3}을 보기 바랍니다.
\R
	\cite{kr1} \S\ A18 \Page{214--219} \\
	\cite{kr2} \S\ A13 \Page{234--239} \\
	\cite{c89} \S\ B.2 \\
	\cite{hs} \Page{423--435} Appendix B
\end{faq}

\begin{faq}
\Q{18.15b}
	C 컴파일러를 테스트해 볼 수 있는 tool들은 없을까요?

\A
A:	Plum Hall (formerly in Cardiff, NJ; now in Hawaii) sells one;
	other packages are Ronald Guilmette's RoadTest(tm) Compiler Test
	Suites (ftp to netcom.com, pub/rfg/roadtest/announce.txt for
	information) and Nullstone's Automated Compiler Performance
	Analysis Tool.
\begin{verbatim}
  http://www.nullstone.com/
\end{verbatim}
	\noindent The FSF's GNU C
	(gcc) distribution includes a c-torture-test which checks a
	number of common problems with compilers.   Kahan's paranoia
	test, found in netlib/paranoia on netlib.att.com, strenuously
	tests a C implementation's floating point capabilities.

\end{faq}


\begin{faq}
\Q{18.15c}
	쓸모 있는 코드나 예제를 모아놓은 것은 없을까요?
\A
	Bob Stout씨의 ``SNIPPETS''는 매우 인기있으며,
	\TT{ftp.brokersys.com}의 \TT{pub/snippets}나 
	\TT{http://www.brokersys.com/snippets/}에 있습니다.

	Lars Wirzenius씨의 `public' 라이브러리는 \TT{ftp.funet.fi}의
        다음 위치에 있습니다:
\begin{verbatim}
  pub/language/C/Publib/
\end{verbatim}
	\noindent \seealso{\ql{14.12}, \ql{18.9}, \ql{18.13}, \ql{18.16}}

\T
	SNIPPETS의 홈페이지는 아래로 변경되었습니다:
\begin{verbatim}
  http://www.snippets.org/
\end{verbatim}


\end{faq}

\begin{faq}
\Q{18.15d}
	% TODO: multiple precison arithmetic을 어떻게 번역하지?
	``multiple precison arithmetic''을 위한 코드가 필요합니다.
\A
	BSD Unix libc 소스에 있는 ``quad'' 함수들은 매우 인기있는 패키지입니다.
	(\TT{ftp.uu.net}의 다음 위치에 있습니다:
\begin{verbatim}
  /systems/unix/bsd-sources/ .... /src/lib/libc/quad/*
\end{verbatim}
	\noindent GNU MP 라이브러리, 또 MIRACL 패키지
	(\TT{http://indigo.ie/~mscott/} 참고), 그리고 오래된 Unix에서
	제공되는 libmp.a를 참고 바랍니다.  \seealso{\ql{14.12}, \ql{18.16}}

\R
	\cite{swsolution} \S\ 17 \Page{343--454}
\end{faq}

\begin{faq}
\Q{18.16}
	이러한 공개용 프로그램들을 어떻게 얻을 수 있나요?
\A
	프로그램 숫자도 갈수록 증가하고, 이런 프로그램들을 정리해 놓은 
	보관(archive) 사이트들도 갈수록 증가하고 있고, 이러한 프로그램들을
	쓰려는 사람들도 덩달아 많아지고 있기 때문에 
	이 질문은 매우 대답하기 쉬우면서도 어렵습니다.

	널리 알려진, 공개를 취지로 한 archive 사이트들을 예로 들면,
	\TT{ftp.uu.net}, \TT{archive.wustl.edu}, \TT{oak.oakland.edu},
	\TT{sumex-aim.stanford.edu}, 그리고 \TT{wuachive.wustl.edu}가
	있습니다.  이 사이트들은 모두 많은 소프트웨어를 가지고 있으며,
	많은 정보들이 공개되어 있습니다.
	FSF의 GNU 프로젝트의 주 사이트는 \TT{prep.ai.mit.edu}입니다.
	이 사이트는 정말로 바쁘기 때문에 접속이 거부당할 수도 있지만,
	그만큼 많은 미러 사이트가 있으므로 참고하기 바랍니다.

	인터넷을 통하여
	이런 사이트에서 파일을 받을 수 있는 가장 일반적인 방법은 
	anonymous ftp를 쓰는 것입니다.  ftp를 쓸 수 없다면, mail로
	ftp를 쓸 수 있게 해주는 ftp-by-mail 서버가 많이 있으므로
	이것을 쓰면 됩니다.  게다가 world-wide web (WWW)는 
	많은 데이터 파일들을 전송하고 인덱싱하고 알리는 데 쓰이고
	있으며, 데이터를 얻는 가장 새로운 방법입니다.

	그리고 이 질문에 대한 답은 쉬운 편도 있지만, 어려운 것은
	세부적인 내용에 대한 것입니다.  즉 쓸 수 있는 archive 사이트들을
	모두 알려주거나 이 사이트에서 데이터를 받아보는 방법을 알려
	주는 것은 불가능합니다.  인터넷에 접속할 수 있다면 여러분은 이
	글보다 훨씬 새롭고 알찬 사이트 및 접근 방법을 이미 알고 있을
	것입니다.

	또 여러분이 찾고자 하는 것이 어디에 있는 지 알려주기도 쉬운
	일이 아닙니다.  이미 이러한 일을 처리할 수 있는 새로운 인덱싱
	서비스가 널려 있습니다.  Alta Vista나 Excite, Yahoo가 그 예입니다.

	유즈넷을 쓸 수 있다면 \TT{comp.sources.unix}나 \TT{comp.sources.misc}
	뉴스 그룹에 정기적으로 게시되는, 보관 정책과 아카이브에 접속하는
	방법을 읽어보시기 바랍니다.  이 그룹의 글을 볼 수 있는 두 개의
	사이트를 소개하면:

	\begin{itemize}
	\item ftp://gatekeeper.dec.com/pub/usenet/comp.sources.unix/
	\item ftp://ftp.uu.net/usenet/comp.sources.unix/
	\end{itemize}

	뉴스 그룹 \TT{comp.archives}은 여러가지 주제에 대한 많은 글을
	게시하고 있습니다.  마지막으로 뉴스 그룹 \TT{comp.sources.wanted}에서는
	소스를 찾고자 할 때 질문을 게시할 수 있는 더 적당한 곳입니다.
	그러나 먼저 이 뉴스 그룹의 FAQ 목록을 읽고 어떻게 소스를 찾을 수
	있는지 알고 나서 글을 게시하기 바랍니다.

	\seealso{\ql{14.12}, \ql{18.13}, \ql{18.15c}}

\T
	FSF의 GNU 프로젝트의 공식 홈페이지는 다음과 같습니다:
\begin{verbatim}
  http://www.gnu.org/
\end{verbatim}

\end{faq}

%
% Local Variables:
% coding: utf-8
% fill-column: 78
% End:
%

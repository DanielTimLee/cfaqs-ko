% $Id$
\chapter{Boolean Expressions and Variables}	\label{chap:boolexpr}
% chapter 9: boolexpr

\begin{rawhtml}
<pre>$Id$</pre>
\end{rawhtml}

% TODO: C99는 이제 _Bool을 지원함. update 바람.
C provides no formal, built-in Boolean type. Boolean values are just integers
(though with greatly reduced range!), so they can be held in any integral type.
C interprets a zero value as ``false'' and \EM{any} nonzero value as ``true.''
The relational and logical operators, such as \TT{==}, \TT{!=}, \verb+<+,
\verb+>=+, \verb+&&+, and \verb+||+, return the value 1 for ``true,'' so
1 is slightly more distinguished as a truth value than the other nonzero
values (but see question \ql{9.2}).

C 언어는 C99 표준 다음부터 이제 boolean 타입을 지원하지만, 문제는 boolean
타입의 필요성에 따라, 표준으로 제정되기 이전부터 개발자들이 임의로 boolean
타입을 만들어 써 왔다는 것입니다. 따라서, 기존 프로그램들과 호환성을 위해서,
boolean 타입의 이름은 (이상하지만) \verb+_Bool+입니다. 좀 더 자세한 것은
질문 \ql{9.1}을 보기 바랍니다.

\begin{faq}
\Q{9.1} % 9.1 COMPLETE
	C 언어에서 불리언(boolean) 값으로 쓸 수 있는 괜찮은 타입이 있을까요?
	그리고 왜 그런 타입이 표준으로 지정되어 있지 않는 거죠?
	참(true)과 거짓(false)을 나타내기 위해 \verb+#define+으로
	정의를 하는 게 좋을까요, 아니면 \TT{enum}을 쓰는 것이 좋을까요?

\A
	C 언어는 표준으로 제공하는 불리언(boolean) 타입이 없습니다.
	왜냐하면, 불리언 타입은 항상 공간/처리시간의 `tradeoff' 문제를
	가지기 때문입니다.  따라서 이는 프로그래머가 알아서 결정하도록
	해 놓은 것입니다.  (\TT{int}를 쓰면 처리가 빠를 수 있고,
	\TT{char}를 쓰면 공간을 절약할 가능성이 있습니다.\footnote{Bitfield를
          쓰면 더 공간을 절약할 가능성이 있습니다. 질문 \ql{2.26} 참고.
          이 경우, unsigned 타입의 bitfield가 필요합니다. 1-bit로 된 signed
          타입 bitfield는 호환성을 보장하면서 +1의 값을 저장할 수 없습니다.}
        그리고, 공간을 절약하기 위해 작은 데이터 타입을 썼다면, 나중에 \TT{int} 
        타입으로 변환하는 코드가 많이 필요할 경우, 오히려 코드의 크기가 커지고
        느려질 수 있다는 단점이 있습니다.)

	\verb+#define+과 \TT{enum} 중 어느 것을 쓸 것인지는 그리 큰 문제가
	되지 않습니다 (질문 \ql{2.22}와 \ql{17.10}을 참고하기 바랍니다.)
	다음 둘 중 아무 것이나 써도 좋으며, 0과 1을 직접 써도 좋습니다:

\begin{verbatim}
  #define TRUE  1             #define YES 1
  #define FALSE 0             #define NO  0

  enum bool {false, true};    enum bool {no, yes};
\end{verbatim}
	\noindent 그러나 일단 어떤 것을 쓰겠다고 정했다면, 한 프로그램이나
	프로젝트 내에서는 계속 그것을 쓰는 것이 좋습니다.  (그러나 디버거에서
	변수의 값을 검사할때, 수치가 아닌 이름으로 보여 줄 수 있으므로
	\TT{enum}이 더 좋을 수도 있습니다. (\note 질문 \ql{2.22} 참고)

	다음과 같이 \TT{typedef}를 쓸 수도 있습니다:
\begin{verbatim}
  typedef int bool;
\end{verbatim}
	\noindent 또는
\begin{verbatim}
  typedef char bool;
\end{verbatim}
	\noindent 또는
\begin{verbatim}
  typedef enum { false, true } bool;
\end{verbatim}


	어떤 사람들은 다음과 같이 쓰는 것을 선호합니다:
\begin{verbatim}
  #define TRUE (1==1)
  #define FALSE (!TRUE)
\end{verbatim}
	\noindent 또는 다음과 같이 ``도우미(helper)'' 매크로를 만듭니다:
\begin{verbatim}
  #define Istrue(e) ((e) != 0)
\end{verbatim}

	그렇지만 이런 방식이 특별히 좋다는 것은 아닙니다.  (질문 \ql{9.2}를
	참고하기 바랍니다; 질문 \ql{5.12}와 \ql{10.2}도 보시기 바랍니다).

\T
	C99에서 \verb+_Bool+ 타입을 제공하기 때문에, 위 설명은 이제 올바른
        답변이 아닙니다. C 언어는 boolean 타입을 제공하며, 타입 이름은
        \verb+_Bool+입니다. 깔끔한 이름이 아닌, 이렇게 이상한 이름을 쓰는
        이유는, 기존의 많은 C 프로그램들이, 위에서 설명한 것처럼 나름대로
        boolean 타입을 이미 쓰고 있기 때문에, 이들 프로그램들과 호환을 위해서
        이러한 이름을 쓰게 되었습니다. 표준 헤더 파일 \verb+<stdbool.h>+를
        포함시키면, 매크로 \TT{bool}\footnote{이미 C++ 언어는 boolean 타입으로
          \TT{bool}이라는 키워드를 제공하기 때문에, 주의가 필요합니다.}
        이 \verb+_Bool+로 정의되어 있기 때문에,
        다음과 같이 쓸 수 있습니다:
\begin{verbatim}
  #include <stdbool.h>

  _Bool a;    /* okay */
  bool  b;    /* okay, using macro `bool' */
\end{verbatim}
	\noindent 또, \verb+<stdbool.h>+는 매크로 \TT{true}, \TT{false}를
        정의하고 있기 때문에 각각 1, 0 대신으로 쓸 수 있습니다.
        
        Boolean 타입이 올 자리에 다른 정수 타입을 쓰게되면, 자동으로 변환되며,
        아시다시피, 0이 아닌 모든 값은 1로, 0은 0으로 바뀝니다.

        또한, C 언어가 boolean 타입을 지원한다고 해서, C++ 처럼, 
        비교 연산자들의 비교 결과가 \verb+_Bool+ 타입인
        것은 아닙니다. 비교 연산자의 비교 결과는 여전히 \TT{int}인 것을
        주의하기 바랍니다. 여기에서 비교 연산자라고 하는 것은,
        표준에서 relational operator와 equality operator를 포함한 것입니다.
        즉, \verb+<+, \verb+<=+, \verb+>+, \verb+>=+, \verb+==+,
        \verb+!=+가 여기에 해당합니다.
\R
	\cite{c89} \S\ 6.2.5 \page{33}, \S\ 6.3.1.2 \page{43}, 
                   \S\ 6.5.8 \page{86}, \S\ 6.5.9 \page{86}
                   \S\ 7.16 \page{252} \\
	\cite{hs5} \S\ 5.1.2 \page{128}, \S\ 5.1.5 \page{132}
                   \S\ 7.6.4 \page{233}
\end{faq}

\begin{faq}
\Q{9.2}
	\verb+#define+을 써서 \TT{TRUE}를 1로 만드는 것은 위험하지 않을까요?
	C 언어에서는 0이 아닌 비트가 하나라도 있으면 참(true)으로 인식될
	테니까요.  만약에 어떤 함수나 연산자가 1이 아닌 다른 값을 리턴할 경우
	어떻게 되는 거죠?

\A
	맞습니다.  사실 C 언어에서는 0이 아닌 모든 값이 참(true)입니다.
	그러나 이것은 단지 ``입력''일 경우에만 의미를
	가집니다; 다시 말하면 어떤 불리언 값이 예상되는 경우에만 그 의미를
	가집니다.  만약 불리언 값이 내장된(built-in) 연산자에 의해 만들어진
	것이라면 분명히 0 또는 1의 값을 가지게 됩니다.  따라서 다음과 같이
	코드를 만들어도 안전합니다 (\TT{TRUE}가 1이라고 가정할 때):
\begin{verbatim}
  if ((a == b) == TRUE)
\end{verbatim}
	그러나 위와 같이 직접 \TT{TRUE}나 \TT{FALSE}와 같은 지 비교하는 것은
        무의미할 뿐더러, 매우 위험합니다.
	왜냐하면 어떤 라이브러리 함수들은 (특히 \TT{isupper()},
	\TT{isalpha()}, 등등) 참(true)을 나타내기 위해 `0이 아닌 값'을 씁니다.
        즉, 이 때 `0이 아닌 값'은 꼭 1일 필요가 없습니다.
        
	(또, 만약 여러분이 아래 코드보다
\begin{verbatim}
  if (a == b)
\end{verbatim}
	\noindent 다음 코드가 더 향상된 것이라고 믿는다면,
\begin{verbatim}
  if ((a == b) == TRUE)
\end{verbatim}
        \noindent 왜 다음과 같이 쓰지 않나요?
\begin{verbatim}
  if (((a == b) == TRUE) == TRUE)
\end{verbatim}
	\noindent 아예 다음과 같이 쓰는게 더 좋은 것이라고 믿는 것인가요?
\begin{verbatim}
  if ((((a == b) == TRUE) == TRUE) == TRUE)
\end{verbatim}
	\noindent Lewis Caroll의 에세이,
        ``What the Tortoise Said to Achilles.''를 읽어보시기 바랍니다.)

        \TT{if (a == b)}가 완벽하게 올바른 것과 마찬가지로, 아래 코드도
        완벽히 올바른 코드입니다:
\begin{verbatim}
  #include <ctype.h>
  ...
  if (isupper(c)) {
    ...
  }
\end{verbatim}
 	\noindent 왜냐하면, \TT{isupper}는, 참/거짓을, zero/nonzero로 알려주기 
        때문입니다. 비슷하게, 아래 코드는 좋은 코드입니다:
\begin{verbatim}
  int isvegetable;   /* really a bool */
  ...
  if (isvegetable) {
    ...
  }
\end{verbatim}
	\noindent 마찬가지로, 아래 코드도 좋습니다:
\begin{verbatim}
  extern int fileexists(char *);  /* returns true/false */
  ...
  if (fileexists(outfile)) {
    ...
  }
\end{verbatim}
	\noindent 위 두 예에서 \TT{isvegetable}과 \TT{fileexists()}는 
        개념상 boolean 타입입니다. 따라서 다음과 같이 쓰는 것은 전혀
        개선이라고 할 수 없습니다:
\begin{verbatim}
  if (isvegetable == TRUE)
\end{verbatim}
	\noindent 또는,
\begin{verbatim}
  if (fileexists(outfile) == YES)
\end{verbatim}
	\noindent (이러한 스타일이 좀 더 ``안전한'' 또는 ``좋은 스타일''이라고
        생각한다면, 정말로 위험하고 나쁜 스타일을 믿고 있다고 말할 수 있습니다.
        가독성도 오히려 떨어집니다. 질문 \ql{17.10}을 참고하기 바랍니다.)

        여기에서 배울 가장 중요한 규칙은,
        값을 대입할 때, 함수의 인자로 전달할 때, 또는 불리언 값을
	리턴할 때에만 \TT{TRUE}나 \TT{FALSE}를 쓴다입니다.
	즉, 비교할 때에 이런 것을 쓰는 것은 좋지 않습니다. (질문 \ql{5.3} 참고)

	\TT{TRUE}나 \TT{FALSE}와 같은 (또는 \TT{YES}와 \TT{NO}) 매크로를
	쓰는 것은 어떻게 보면 좋은 것 같지만 C 언어에서 불리언 값이라는
	개념이 헷갈리기 때문에, 어떤 프로그래머들은
	0과 1이라는 수치를 직접 쓰는 것을 선호하기도 합니다.  (질문 \ql{5.9}를
	참고하기 바랍니다.)

\R
	\cite{kr1} \S\ 2.6 \page{39}, \S\ 2.7 \page{41} \\
	\cite{kr2} \S\ 2.6 \page{42}, \S\ 2.7 \page{44}, 
		\S\ A7.4.7 \page{204}, \S\ A7.9 \page{206} \\
	\cite{c89} \S\ 6.3.3.3, \S\ 6.3.8, \S\ 6.3.9, \S\ 6.3.13, 
		\S\ 6.3.14, \S\ 6.3.15, \S\ 6.6.4.1, \S\ 6.6.5 \\
	\cite{hs} \S\ 7.5.4 \Page{196--7}, \S\ 7.6.4 \Page{207--8}, 
		\S\ 7.6.5 \Page{208--9}, \\
        \cite{hs} \S\ 7.7 \Page{217--8},
		\S\ 7.8 \Page{218--9}, \S\ 8.5 \Page{238--9}, 
		\S\ 8.6 \Page{241--4} 
	Carroll, ``What the Tortoise Said to Achilles''
\T
	C99는 표준 헤더 파일, \verb+<stdbool.h>+을 통해서 매크로
        \TT{true}와 \TT{false}를 제공하며, 각각 0과 1로 정의되어 있습니다.
	질문 \ql{9.1}에서 새 boolean type인 \verb+_Bool+을 참고하기 바랍니다.
\end{faq}

\begin{faq}
\Q{9.3} % 9.3 COMPLETE
	\TT{p}가 포인터일때, \verb+if (p)+와 같은 코드를 쓰는 것은
	나쁜가요?
\A
	아닙니다.  좋습니다.  질문 \ql{5.3}을 참고하기 바랍니다.
\end{faq}

\begin{faq}
\Q{9.4} % 9.4 COMPLETE
	\TT{TRUE}나 \TT{FALSE}와 같은 심볼을 쓰는 것이 좋을까요, 아니면,
        1과 0을 직접 쓰는 것이 좋을까요?
\A
	그것은 여러분이 결정할 문제입니다.
	Preprocessor 매크로인 \TT{TRUE}, \TT{FALSE}는 (물론
	\TT{NULL} 포함) 단지 코드를 읽기 쉽게 하기 위해 쓰는 것이지,
	이들 매크로가 나타내는 값이 바뀔 가능성이 있기 때문은
	아닙니다. (질문 \ql{5.10}, \ql{17.10} 참고)
        즉 이것은 스타일에 관한 문제이지 옳고 그른 문제가 아닙니다.

        어떤 사람들은 \TT{TRUE}와 \TT{FALSE}같은 심볼 이름을 쓰는 것은, 
        코드를 읽는 사람에게 boolean 값이 쓰인다는 것을 알려주는 역할을 
        하기 때문에,
        좋다고 합니다. 또 어떤 사람들은 어차피 C 언어에서 boolean 값과 정의가
        혼동스럽기 때문에, \TT{TRUE}와 \TT{FALSE}같은 매크로를 쓰는 것은 오히려
        더 복잡하게 만든다고 생각합니다. (질문 \ql{5.9} 참고)
\end{faq}


\begin{faq}
\Q{9.5} % 9.5 COMPLETE 
	다른 곳에서 라이브러리를 받아 왔는데, 그 라이브러리에서 이미 \TT{TRUE}와
        \TT{FALSE}를 다르게 정의하고 쓰고 있어서, 제 코드와 충돌이 일어납니다.

\A
	질문 \ql{10.10}을 보기 바랍니다.
\end{faq}

%
% Local Variables:
% coding: utf-8
% fill-column: 78
% End:
%
